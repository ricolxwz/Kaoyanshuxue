\chapter{预备知识}
\section{基础知识}
\subsection{函数概念}
\subsubsection{函数}
\subsubsection{反函数}
\begin{itemize}
	\item 严格单调的函数必有反函数, 有反函数的函数不一定是单调函数
    \item $ x=f^{-1}(y) $和$ y=f(x) $是同一个函数
\end{itemize}
\subsubsection{复合函数}
\subsubsection{四种特性}
设$ D $为定义域
\begin{itemize}
	\item 单调性: $ \forall x_{1}, x_{2}\in I \in D $, $ x_{1}<x_{2}, f(x_{1})>f(x_{2}) $, 单调递减
	\item 有界性: $ \forall x\in I \in D $, $ \exists M>0 $, $ |f(x)|\le M $
	\item 奇偶性: $ D $关于原点对称, $ \forall x\in D $, $ f(x)=f(-x) $, 偶函数
	\item 周期性: $ x\in D, x\pm T\in D $, $ \exists T>0 $, $ f(x+T)=f(x) $
\end{itemize}
函数和导函数:
\begin{itemize}
	\item $ f(x) $为可导偶函数$ \Rightarrow f'(x)$为奇函数\\$ f(x) $为可导奇函数$ \Rightarrow f'(x)$为偶函数
	\item $ f(x) $为可导周期为$ T $函数$ \Rightarrow f'(x) $为周期为$ T $函数
	\item $ f(x) $在$ (a,b) $可导且$ f'(x) $有界$ \Rightarrow f(x) $在$ (a,b) $有界
\end{itemize}
原函数和积分:
\begin{itemize}
	\item 连续的奇函数的一切原函数为偶函数\\连续的偶函数中仅有一个原函数为奇函数
	\item 连续函数周期为$ T $且$ \int_{0}^{T}f(x)dx=0 $$ \Rightarrow $原函数周期为$ T $
\end{itemize}
\subsection{函数图像}
\subsubsection{直角坐标系}
常见图像:
\begin{itemize}
	\item 三角函数
	\begin{itemize}
		\item 正弦函数和余弦函数
		\item 正切函数和余切函数
		\item 正割函数和余割函数
	\end{itemize}
    \item 反三角函数
    \begin{itemize}
    	\item 反正弦函数和反余弦函数
    	\item 反正切函数和反余切函数
    \end{itemize}
    \item 取整函数
\end{itemize}
图像变换:
\begin{itemize}
	\item 平移变换
	\begin{itemize}
		\item 水平平移
		\item 垂直平移
	\end{itemize}
	\item 对称变换
	\begin{itemize}
		\item $ x $轴对称变换
		\item $ y $轴对称变换
		\item 原点对称变换
		\item $ y=x $对称变换
	\end{itemize}
	\item 伸缩变换
	\begin{itemize}
		\item 水平伸缩变换
		\item 垂直伸缩变换
	\end{itemize}
\end{itemize}
\subsubsection{极坐标系}
用描点法画常见图像:
\begin{itemize}
	\item 心形线: $ r=a(1-\cos \theta)\ (a>0) $
	\item 玫瑰线: $ r=a\sin 3\theta\ (a>0) $
	\item 阿基米德螺旋线: $ r=a\theta\ (a>0,\theta \ge 0) $
	\item 伯努利双纽线: $ r^{2}=a^{2}\cos 2\theta\ (a>0) $
\end{itemize}
\section{解题技巧}
\subsection{函数概念}
\subsubsection{四种特性}
判断单调性主要有两种方法:
\begin{enumerate}
	\item 求导
	\item 定义法: 单调递减$ \Leftrightarrow (x_{1}-x_{2})(f(x_{1})-f(x_{2}))<0 $
\end{enumerate}


















