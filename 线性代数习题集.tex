\documentclass[oneside, onecolumn]{ctexbook}
\usepackage[UTF8]{ctex}
\usepackage{indentfirst}
\usepackage{bm}
\usepackage{amsmath}
\setlength{\parindent}{0em}

\begin{document}
\chapter{特征值和特征向量}
\section{基础知识}
\subsection{特征值和特征向量}
\subsubsection{性质}
\begin{enumerate}
	\item 特征值的性质
	\item 特征向量的性质
	\begin{enumerate}
		\item $ k $重特征值$ \Lambda $至多只有$ k $个线性无关的向量
		\item 若$ \bm{\xi_1} $, $ \bm{\xi_2} $是$ \bm{A} $的属于不同特征值的特征的特征向量, 则$ \bm{\xi_1}, \bm{\xi_2} $线性无关
		\item 若$ \bm{\xi_1}, \bm{\xi_2} $是$ \bm{A} $的属于同一特征值$ \lambda $的特征向量, 则$ k_1\bm{\xi_1} + k_1\bm{\xi_2} $仍然是$ \bm{A} $的属于特征值$ \lambda $的特征向量
	\end{enumerate}
\end{enumerate}
\subsection{矩阵的相似对角化}
\subsubsection{定义}
设$ n $阶矩阵$ \bm{A} $, 存在$ n $阶可逆矩阵$ \bm{P} $, 使得$ \bm{P}^{-1}\bm{A}\bm{P}=\bm{\Lambda} $, 则$ \bm{A}\sim \bm{\Lambda} $, $ \bm{\Lambda} $是$ \bm{A} $的相似标准形.\par 
\[ \bm{P}=\left[\bm{\xi_1}, \bm{\xi_2},... \bm{\xi_n}\right], \bm{\Lambda} =\begin{bmatrix}
	\lambda_1 &  &  &  \\
	& \lambda_2 &  &  \\
	&  & \ddots &  \\
	&  &  & \lambda_n 
\end{bmatrix}
 \]
\subsubsection{条件}
\begin{enumerate}
	\item $ n $阶矩阵$ \bm{A} $可以相似对角化$ \Leftrightarrow $$ \bm{A} $有$ n $个线性无关的特征向量($ \left| \bm{P}\right| = 0$)
	\item $ n $阶矩阵$ \bm{A} $可以相似对角化$ \Leftrightarrow $$ \bm{A} $对应于每个$ k_i $重特征值都有$ k_i $个线性无关的特征向量($ n $重特征值对应的解空间是$ n $维)
	\item $ n $阶矩阵$ \bm{A} $有$ n $个不同特征值$ \Rightarrow $$ \bm{A} $可以相似对角化(由特征向量的性质3可以推出)
	\item $ n $阶矩阵$ \bm{A} $为实对称矩阵$ \Rightarrow $$ \bm{A} $可以相似对角化
\end{enumerate}\par
上述总共两个充要条件, 两个充分条件.
\subsection{实对称矩阵}
\subsubsection{定义}
若$ \bm{A}^T = \bm{A} $, 则$ \bm{A} $为是对称矩阵, 如果在此基础上$ \bm{A} $的元素都是实数, 则$ \bm{A} $是实对称矩阵.
\subsubsection{性质}
\begin{enumerate}
	\item 实对称矩阵$ \bm{A} $的属于不同特征值的特征向量相互正交
	\item 实对称矩阵$ \bm{A} $必相似于对角矩阵, 且存在正交矩阵$ \bm{Q} $, 使得$ \bm{Q}^{-1}\bm{A}\bm{Q}=\bm{Q}^{T}\bm{A}\bm{Q}=\bm{\Lambda}$,  
\end{enumerate}
\section{习题}
\subsection{实对称矩阵}
\subsubsection{求正交矩阵Q}
\begin{enumerate}
	\item 求$ \bm{A} $的$ \lambda $与$ \bm{\xi} $
	\item $ \bm{\xi_1}, \bm{\xi_{2}},... ,\bm{\xi_{n}} $施密特正交化, 单位化至$ \bm{\eta_{1}}, \bm{\eta_{2}},... ,\bm{\eta_{n}} $
	\item 令$ Q=(\bm{\eta_1}, \bm{\eta_2},... ,\bm{\eta_{n}}) $
\end{enumerate}
\par 不同的特征值$ \lambda_{i} $对应的特征矩阵$ \bm{\xi_{i}} $之间是正交的.
\par 施密特正交化:$ \beta_{1}=\alpha_{1}, 
\beta_{2}=\alpha_{2}-\frac{(\alpha_{2}, \beta_{1})}{(\beta_{1}, \beta_{1})}\beta_{1} $.
\par 单位化: $ \eta_{1}=\frac{\beta_{1}}{||\beta_{1}||} $.
\paragraph{总结} 
\begin{enumerate}
	\item 普通矩阵$ \bm{A} $
	\begin{enumerate}
		\item $ \lambda_{1}\neq \lambda_{2}\Rightarrow \xi_{1}, \xi_{2}$无关
		\item $ \lambda_{1}= \lambda_{2}\Rightarrow \xi_{1}, \xi_{2}$
		\begin{enumerate}
			\item $\xi_{1}, \xi_{2}$无关
			\item $\xi_{1}, \xi_{2}$相关
		\end{enumerate}
	\end{enumerate}
    \item 实对称矩阵$ \bm{A} $
    \begin{enumerate}
    	\item $ \lambda_{1}\neq \lambda_{2}\Rightarrow \xi_{1}\perp \xi_{2}$, $\xi_{1}, \xi_{2}$无关
    	\item $ \lambda_{1}= \lambda_{2}\Rightarrow$ 
    	\begin{enumerate}
    		\item $\xi_{1}\perp \xi_{2}$, $\xi_{1}, \xi_{2}$无关
    		\item $\xi_{1}$不垂直于$\xi_{2}$, $\xi_{1}, \xi_{2}$无关
    	\end{enumerate}
    \end{enumerate}
\end{enumerate}
















\end{document} 