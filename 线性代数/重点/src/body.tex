\chapter{行列式}
\begin{enumerate}
\item $k \times
\begin{vmatrix}
a_{11} & a_{12} & \dots & a_{1n} \\
a_{21} & a_{22} & \dots & a_{2n} \\
\vdots & \vdots & \ddots & \vdots \\
a_{n1} & a_{n2} & \dots & a_{nn}
\end{vmatrix} =
\begin{vmatrix}
ka_{11} & ka_{12} & \dots & ka_{1n} \\
a_{21} & a_{22} & \dots & a_{2n} \\
\vdots & \vdots & \ddots & \vdots \\
a_{n1} & a_{n2} & \dots & a_{nn}
\end{vmatrix} $
与矩阵不同, 牢记!
\end{enumerate}
\chapter{矩阵}
\begin{enumerate}
\item $|\bm{A}^{-1}|=|\bm{A}|^{-1}$
\item $\bm{E}_{31}(k)$的含义是第$1$行的$k$倍加到第$3$行, 或者是第$3$列的$k$倍加到第$1$列
\item $|k \bm{A}|=k^{n}|\bm{A}|$
\item $(k \bm{A})^{*}=k^{n-1} \bm{A}^{*}$
\item $|\bm{A}|\neq 0\Leftrightarrow$满秩$\Leftrightarrow \bm{A}$可逆
\end{enumerate}
\chapter{向量}
\begin{enumerate}
\item 判别线性相关性的七大定理:
\begin{enumerate}
\item 充分必要条件(两个):
\begin{enumerate}
\item $\bm{A}\bm{x}=\bm{0}$, $\bm{x}$有非零解
\item 向量组中至少有一个向量能够被其他的向量线性表出
\end{enumerate}
\item 与解线性方程组有关的定理(两个):
\begin{enumerate}
\item $\Rightarrow$: 如果向量组$\bm{{\alpha_{1}}},\bm{{\alpha_{2}}},...,\bm{{\alpha_{n}}}$线性无关, 而向量组$\bm{{\alpha_{1}}},\bm{{\alpha_{2}}},...,\bm{{\alpha_{n}},\bm{\beta}}$线性相关, 则$\bm{\beta}$能由向量组$\bm{{\alpha_{1}}},\bm{{\alpha_{2}}},...,\bm{{\alpha_{n}}}$线性表示, 且表示方法唯一
\item $\Leftarrow$: 如果$\bm{\beta}$能由向量组$\bm{{\alpha_{1}}},\bm{{\alpha_{2}}},...,\bm{{\alpha_{n}}}$线性表示, 则非齐次线性方程组$\bm{A}\bm{x}=\bm{\beta}$有解, 且$r([\bm{{\alpha_{1}}},\bm{{\alpha_{2}}},...,\bm{{\alpha_{n}}}])=r([\bm{{\alpha_{1}}},\bm{{\alpha_{2}}},...,\bm{{\alpha_{n}}},\bm{\beta}])$
\end{enumerate}
\item 整体性和局部性(两个):
\begin{enumerate}
\item 向量 \par
如果低维是无关的, 则增加维数必无关\par
如果高维是相关的, 则减少维数必相关
\item 向量组 \par
如果整体是无关的, 则部分必无关 \par
如果部分是相关的, 则整体必相关
\end{enumerate}
\item 以少表多, 多的相关 \par
如果向量组$\bm{\alpha_{1}},\bm{\alpha_{2}},...\bm{\alpha_{s}}$能被向量组$\bm{\beta_{1}},\bm{\beta_{2}},...\bm{\beta_{t}}$线性表示, 且$t<s$, 则向量组$\bm{\alpha_{1}},\bm{\alpha_{2}},...\bm{\alpha_{s}}$线性相关
\end{enumerate}
\item 设有向量组$\bm{\alpha_{1}},\bm{\alpha_{2}},...\bm{\alpha_{s}}$和向量组$\bm{\beta_{1}},\bm{\beta_{2}},...\bm{\beta_{t}}$, 若后者的所有元素都能被前者线性表出, 则:
\begin{equation*}
r([\bm{\beta_{1}},\bm{\beta_{2}},...\bm{\beta_{t}}])\le r([\bm{\alpha_{1}},\bm{\alpha_{2}},...\bm{\alpha_{s}}])
\end{equation*}
\item $\bm{A}$经过初等变换得到了$\bm{B}$, 则:
\begin{enumerate}
\item $\bm{A}$和$\bm{B}$的相应部分的列向量具有相同的线性相关性(这条性质为后面解线性方程打下基础)
\item $\bm{A}$的行向量组和$\bm{B}$的行向量组是等价向量组(可以互相线性表示)
\end{enumerate}
\end{enumerate}
\chapter{线性方程组}
\begin{enumerate}
\item 齐次线性方程组:
\begin{enumerate}
\item $m>n\Rightarrow $有非零解
\item $m=n$:
\begin{enumerate}
\item $r(\bm{A})=m\Rightarrow$满秩$\Rightarrow |\bm{A}|\neq 0 \Rightarrow$线性无关$\Rightarrow$唯一零解
\item $r(\bm{A})=r<m\Rightarrow$不满秩$\Rightarrow |\bm{A}=0\Rightarrow$线性相关$\Rightarrow$有非零解, 且有$m-r$个线性无关解(也可以换一种解释: 由于$r(\bm{A})=r<m\Rightarrow$独立方程组的个数为$r$, 由于$r<m$, 所以有非零解
\end{enumerate}
\end{enumerate}
\item 非齐次线性方程组
\begin{enumerate}
\item $r(\bm{A})\neq r([\bm{A},\bm{b}])\Rightarrow$无解
\item $r(\bm{A})=r([\bm{A},\bm{b}])=m\Rightarrow$唯一解
\item $r(\bm{A})=r([\bm{A},\bm{b}])=r<m\Rightarrow$无穷解
\end{enumerate}
\end{enumerate}
\chapter{特征值和特征向量}
\begin{enumerate}
\item 特征值的性质:
\begin{enumerate}
\item 特征值的数量为$n$ (包括重根)
\item $\sum_{i=1}^{n}\lambda_{i}=tr(\bm{A})$
\item $\prod_{i=1}^{n}\lambda_{i}=|\bm{A}|$
\end{enumerate}
\item 特征向量的性质:
\begin{enumerate}
\item 线性无关的特征向量的数量$\le n$
\item 每个特征值至少有一个特征向量
\item $k$重特征值至多有$k$个线性无关的特征向量
\item 若$\bm{\xi_{1}},\bm{\xi_{2}}$是属于不同特征值$\lambda_{1},\lambda_{2}$的特征向量, 则$\bm{\xi_{1}},\bm{\xi_{2}}$线性无关
\item 若$\bm{\xi_{1}},\bm{\xi_{2}}$是属于同一特征值$\lambda$的特征向量, 则$k_{1}\bm{\xi_{1}}+k_{2}\bm{\xi_{2}}$仍然是特征值$\lambda$的特征向量
\end{enumerate}
\item 矩阵相似的性质(假设$\bm{A}\sim \bm{B}$)
\begin{enumerate}
\item 反身性, 对称性, 传递性
\item 相等
\begin{enumerate}
\item $r(\bm{A})=r(\bm{B})$
\item $|\bm{A}|=|\bm{B}|$
\item 特征值相等
\item 特征多项式相等
\end{enumerate}
\item 函数(注意是相似不是相等)
\begin{enumerate}
\item $f(\bm{A})\sim f(\bm{B})$
\item $\bm{A^{T}}\sim \bm{B^{T}}$
\item 若$\bm{A}$为可逆矩阵, $\bm{A^{-1}}\sim \bm{B^{-1}}$
\item 若$\bm{A}$为可逆矩阵, $\bm{A^{*}}\sim \bm{B^{*}}$
\end{enumerate}
\end{enumerate}
\item 相似对角矩阵的条件
\begin{enumerate}
\item 充要条件
\begin{enumerate}
\item 矩阵$\bm{A}$有$n$个线性无关的特征向量
\item $n$维特征值对应$n$维解空间
\end{enumerate}
\item 充分条件
\begin{enumerate}
\item 矩阵$\bm{A}$有$n$个不同的特征值
\item 矩阵$\bm{A}$为实对称矩阵
\end{enumerate}
\end{enumerate}
\item 实对称矩阵的性质
\begin{enumerate}
\item 若$\bm{\xi_{1}},\bm{\xi_{2}}$是属于不同特征值$\lambda_{1},\lambda_{2}$的特征向量, 则$\bm{\xi_{1}},\bm{\xi_{2}}$正交
\item 实对称矩阵一定相似于对角阵
\item 存在可逆矩阵$\bm{P}$, 使$\bm{P}\bm{A}\bm{P^{-1}}=\bm{\Lambda}$
\item 存在正交矩阵$\bm{Q}$, 使$\bm{Q^{-1}}\bm{A}\bm{Q}=\bm{Q^{T}}\bm{A}\bm{Q}=\bm{\Lambda}$
\end{enumerate}
其中, $\bm{\Lambda}$和$\bm{P}$的定义和对角矩阵里面的$\bm{\Lambda}$和$\bm{P}$相同
\end{enumerate}
\chapter{二次型}
\begin{enumerate}
\item 二次型
\begin{enumerate}
\item 代数表达形式
\begin{equation*}
\begin{aligned}
f(x) = a_{11}x_{1}^{2}+a_{12}x_{1}x_{2}+a_{13}x_{1}x_{3}...+a_{1n}x_{1}x_{n} & \\
+a_{22}x_{2}^{2}+a_{23}x_{2}x_{3}+...+a_{2n}x_{2}x_{n} & \\
+\dots & \\
+a_{nn}x_{n}^{2} & \\
\end{aligned}
\end{equation*}
\item 矩阵表达形式
\begin{equation*}
f(\bm{x})=\bm{x}^{T}\bm{A}\bm{x},
\bm{A}=\begin{bmatrix}
a_{11} & a_{12} & \dots & a_{1n} \\
a_{21} & a_{22} & \dots & a_{2n} \\
\vdots & \vdots &  & \vdots \\
a_{n1} & a_{n2} & \dots & a_{nn}
\end{bmatrix},
\bm{x}=
\begin{bmatrix}
x_{1} \\
x_{2} \\
\vdots \\
x_{n} \\
\end{bmatrix}
\end{equation*}
\end{enumerate}
\item 合同
\begin{enumerate}
\item (可逆)线性变换 \par
对于$ n $元二次型$ f(x_{1}, x_{2},... ,x_{n}) $, 若令
\begin{equation*}
\left\{
\begin{aligned}
& x_{1} = c_{11}y_{1}+c_{12}y_{2}+\dots +c_{1n}y_{n}, \\
& x_{2} = c_{21}y_{1}+c_{22}y_{2}+\dots +c_{2n}y_{n}, \\
& \dots \\
& x_{n} = c_{n1}y_{1}+c_{n2}y_{2}+\dots +c_{nn}y_{n},
\end{aligned}
\right.
\end{equation*}\par
记$ \bm{x}=\begin{bmatrix}
x_1 \\
x_2 \\
\vdots \\
x_n
\end{bmatrix}, \bm{C}=\begin{bmatrix}
c_{11} & c_{12} & \dots & c_{1n} \\
c_{21} & c_{22} & \dots & c_{2n} \\
\vdots & \vdots &  & \vdots \\
c_{n1} & c_{n2} & \dots & c_{nn}
\end{bmatrix}, \bm{y}=\begin{bmatrix}
y_1 \\
y_2 \\
\vdots \\
y_n
\end{bmatrix}$\par \vspace{1em}
则上式可以写为$ \bm{x}=\bm{C}\bm{y} $. 上式称为从$ y_{1}, y_{2},... ,y_{n} $到$ x_{1}, x_{2},... ,x_{n} $的线性变换. 如果$ \bm{C} $可逆, 则称为可逆线性变换.\par
\item 合同 \par
我们可以把上式$\bm{x}=\bm{C}\bm{y}$代入到$f(\bm{x})=\bm{x}^{T}\bm{A}\bm{x}$, 令$\bm{B}=\bm{C}^{T}\bm{A}\bm{C}$, 可以得到$f(\bm{y}=\bm{y}^{T}\bm{B}\bm{y})$. 其中$\bm{A}$与$\bm{B}$合同, 记为$\bm{A}\simeq \bm{B}$\ (与相似不同的是这里是转置而不是取逆) \par
合同的几个性质:
\begin{enumerate}
\item 反身性, 对称性, 传递性
\item 可逆线性变化不改变矩阵的秩
\item 与对称矩阵合同的矩阵一定也是对称矩阵
\end{enumerate}
\end{enumerate}
\item 标准形和规范形 \par
标准形是二次型中非二次项的系数全为$0$, 规范形是在标准形的基础上规定二次型的系数都为$1$, $0$或者$-1$.\par
化为标准形有两种方法:
\begin{enumerate}
\item 配方法
\begin{equation*}
\bm{x}=\bm{C}\bm{y}, \bm{C}^{T}\bm{A}\bm{C}=\bm{\Lambda}
\end{equation*}\par
其中, $\bm{\Lambda}$的形式为:
\begin{equation*}
\bm{\Lambda}=\begin{bmatrix}
d_1 &  &  &  \\
& d_2 &  &  \\
&  & \ddots &  \\
&  &  & d_n
\end{bmatrix}
\end{equation*}\par
注意, 这里不一定是特征值, 不是你以为的$\bm{\Lambda}$.
\item 利用实对称矩阵的性质 \par
由于实对称矩阵一定存在一个正交矩阵$\bm{Q}$, 使得$\bm{Q}^{T}\bm{A}\bm{Q}=\bm{Q}^{-1}\bm{A}\bm{Q}=\bm{\Lambda}$, 故:
\begin{equation*}
\bm{x}=\bm{Q}\bm{y}, \bm{Q}^{T}\bm{A}\bm{Q}=\bm{\Lambda}
\end{equation*}
其中, $\bm{\Lambda}$的形式为:
\begin{equation*}
\bm{\Lambda}=\begin{bmatrix}
\lambda_1 &  &  &  \\
& \lambda_2 &  &  \\
&  & \ddots &  \\
&  &  & \lambda_n
\end{bmatrix}
\end{equation*}\par
可以将上述标准形进一步操作得到规范形.
\end{enumerate}
\item 惯性定理 \par
将二次型合同标准化后得到的标准形中, 正数项的个数设为$p$, 负数项的个数设为$q$, 有如下性质:
\begin{enumerate}
\item $r(\bm{A})=p+q$
\item 两个矩阵合同的条件是具有相同的惯性指数
\end{enumerate}
\item 正定二次型 \par
对于任一$\bm{x}\neq \bm{0}$, 有$\bm{x}^{T}\bm{A}\bm{x}>0$, 称$f$为正定二次型, $\bm{A}$为正定矩阵.
\begin{enumerate}
\item 充要条件:
\begin{enumerate}
\item 对于任一$\bm{x}\neq \bm{0}$, 有$\bm{x}^{T}\bm{A}\bm{x}>0$
\item 存在可逆矩阵$\bm{D}$, 使$\bm{A}=\bm{D}^{T}\bm{D}$
\item $f$的正惯性指数为$n$
\item $\bm{A}\simeq \bm{E}$
\item $\bm{A}$的特征值$\lambda_{i}>0$
\item $\bm{A}$的顺序主子式都$>0$
\end{enumerate}
\item 必要条件
\begin{enumerate}
\item $|\bm{A}|>0$
\item $a_{ii}>0$
\end{enumerate}
\end{enumerate}
\end{enumerate}